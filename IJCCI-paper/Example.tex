\documentclass[a4paper,twoside]{article}

\usepackage{epsfig}
\usepackage{subfigure}
\usepackage{calc}
\usepackage{amssymb}
\usepackage{amstext}
\usepackage{amsmath}
\usepackage{amsthm}
\usepackage{multicol}
\usepackage{pslatex}
\usepackage{apalike}
\usepackage{SCITEPRESS}     % Please add other packages that you may need BEFORE the SCITEPRESS.sty package.

\usepackage[tableposition=top]{caption}
\usepackage[hidelinks]{hyperref}
\usepackage{lineno,amsmath,algorithm,algpseudocode}
\modulolinenumbers[1]
\usepackage{graphicx}
\usepackage{caption}
\usepackage{subcaption}
% \usepackage{subfig}
\usepackage{booktabs}
\newcommand{\head}[1]{\textnormal{\textbf{#1}}}

\graphicspath{ {images/} }

\subfigtopskip=0pt
\subfigcapskip=0pt
\subfigbottomskip=0pt

\begin{document}

\title{Authors' Instructions  \subtitle{Preparation of Camera-Ready Contributions to SCITEPRESS Proceedings} }

\author{\authorname{First Author Name\sup{1}, Second Author Name\sup{1} and Third Author Name\sup{2}}
\affiliation{\sup{1}Institute of Problem Solving, XYZ University, My Street, MyTown, MyCountry}
\affiliation{\sup{2}Department of Computing, Main University, MySecondTown, MyCountry}
\email{\{f\_author, s\_author\}@ips.xyz.edu, t\_author@dc.mu.edu}
}

\keywords{The paper must have at least one keyword. The text must be set to 9-point font size and without the use of bold or italic font style. For more than one keyword, please use a comma as a separator. Keywords must be titlecased.}

\abstract{Operational maturity of biological control systems have fuelled the inspiration for a large number of mathematical and logical models for control, automation and optimisation. The human brain represents the most sophisticated control architecture known to us and is a central motivation for several research attempts across various domains. In the present work, we introduce an algorithm for mathematical optimisation that derives its intuition from the hierarchical and distributed operations of the human motor system. The system comprises global leaders, local leaders and an effector population that adapt dynamically to attain global optimisation via a feedback mechanism coupled with the structural hierarchy. The hierarchical system operation is distributed into local control for movement and global controllers that facilitate gross motion and decision making. We present our algorithm as a variant of the classical Differential Evolution algorithm, introducing a hierarchical crossover operation. The discussed approach is tested exhaustively on standard test functions as well as the CEC 2017 benchmark. Our algorithm significantly outperforms various standard algorithms as well as their popular variants as discussed in the results.}

\onecolumn \maketitle \normalsize \vfill

\section{\uppercase{Introduction}}
\label{sec:introduction}

Operational maturity of biological control systems have enamored researchers across various domains. Consequently, they have been the source of inspiration of mathematical and logical models for control, automation and optimization. At the cellular and organ levels, In the E.Coli bacterium, there is sensing and locomotion involved in seeking nourishment and avoiding harmful chemicals. There is sensing via the recognition of chemicals, internal decision making and actuation via locomotion. These behavioural characteristics have fueled the inspiration for the Bacterial Foraging Optimization Algorithm.\\

The function of human brain is at the pinnacle of relevance at the current social and technological standing, and several research initiatives have been attempting to mimic human-like precision, accuracy and efficiency. The brain has developed amazing performance in several day-to-day tasks such as grasping, walking etc. which happens due to the parallel work and management of brain sections governing various steps involved in completion of a task. The brain function activities can be categorised in 2 broad categories: sensory and motor operations. Sensory cortical functions have inspired the concepts of neural networks and deep learning which have already been successfully scaled to solve a vast amount of problems.\\
While not always the case, it is at times useful and accurate to view a biological neural system to be arranged in a hierarchical fashion. One part of the brin that is clearly hierarchical is the human motor system. It is a hierarchical and distributed control system. It can be classified as having local control functions for movement and higher level controllers that control gross motion and decision making involved in planning actions. It is connected to many parts of the brain, so that we can plan, learn and execute motions or actions. The motor learning sequence of operations involves an adaptive model building in attaining optimal co-ordination of motion. The hierarchy of motor operations can be represented as the following steps:
\begin{itemize}
	\item[1.] Motivation and planning of the action.
	\item[2.] Generation of instructions for movement.
	\item[3.] Refinement of instructions based on feedback.
	\item[4.] Maintenance of posture and smooth execution of actions.
\end{itemize}
Signal detection in the somatosensory cortex area, that pertains to touch sense, can participate as a possible motivator for the initiation of motor actions. Motor control is a type of neural hierarchical distributed learning control system. The neurons interface via special cells to the sensory inputs and are also control systems so that we can move our arms and legs via the neuronal control of muscle contraction and expansioin.

The human motor function represents a distributed and hierarchical system. The optimal execution involves distributed brain structures at different levels of hierarchy. These include the pre-frontal cortex, motor cortex, spinal cord, the anterior horn cells etc. in generating action sequences, a sequence of actions is implemented by a string of subsequences of actions, each possibly implemented in a different part of the body. The optimization of the motor learning procedure takes place at multiple levels of hierarchy:

\begin{figure}
  \includegraphics[scale=0.6]{motorControl}
  \caption{Description of Heirarchical Motor Operations in Humans}
  \label{fig:motorControl}
\end{figure}


\begin{itemize}
\item[1.] Stage 1 involves motivation and planning for the action to be performed. This may be categorised as involving collaboration of the sensory and the pre-motor components. For instance, signals in the somatosensory cortex area, pertaining to the sense of touch may trigger a chain of motor actions. For voluntary motor actions, decision making and planning occurs in the pre-frontal cortex and the  pyramidal cells of the motor cortex.

\item[2.] These global leaders then pass the information to the anterior horn cells in the spinal cord. These cells represent the local leaders and govern the operations of a set of effectors.

\item[3.] For optimality of all actions, neurons act in unison. The neurons in the motor cortex act like global leaders and send inhibitory influence over the anterior horn cells (local leaders) located in the spinal cord. These local leaders are connected to the muscle fibre (effectors) through a peripheral nerve and neuromuscular junction.

\item[4.] Whenever, a person plans to perform an action, the electrical signals generated from the pyramidal neurons in the motor cortex (representing the decision making process) are transmitted via a supraspinal tract on the anterior horn cells (communicating to the local leaders). Initially, send inhibitory influence to the population over the local leaders. Efficient execution of task requires feedback based faciliation and inhibition of the effectors. These result in contraction of the antagonist and relaxation of the agonist muscle fibres through the local leaders.

\item[5.] This sequence of updation of the constituent particles characteristics, categories Stage 4, representing the optimaal convergence of the system leading to smooth motor execution (enjoy the coffee!).
\end{itemize}

The prefrontal cortex and motor cortex combined, represent the frontal cortex and constitute the action based decision making process, or in conjunction, what we call the global leader.

\section{Classical Differential Evolution}

The classical Differential Evolution (DE) algorithm is a population-based global optimization algorithm, utilizing a crossover and mmutation approach to generate new individuals in the population for achieving optimum solutions. For each individual $x_i$ that belongs to the population, DE randomly samples three other individuals from the population $a_i$, $b_i$ and $c_i$. Then using the randomly chosen points, a new individual vector is generated using equation \eqref{de}:

\begin{equation}
\label{de}
u_i = a_i + F (b_i - c_i)
\end{equation}

Where, $F$ is called the differential weight (Usually lies between $[0, 1]$).\\
And to obtain the new position of the individual, a crossover operation is implemented between $x_i$ and $u_i$, controlled by the parameter $CR$ called the crossover probability. The value for $CR$ always lies between $[0, 1]$.

\begin{figure}
  \includegraphics[scale=0.3]{contourDE}
  \caption{Two dimensional example of an objective function showing its contour lines and the process for
generating v in scheme DE2}
  \label{fig:contourDE}
\end{figure}

\section{Distributed Leader Optimization}
Taking inspiration from the human motor system, we model the hierarchical motor operations in our optimization agents, where we define a global leader which influences the action of several distributed local leaders and the particle agents which act as the effectors. The global leader is analogous to the decision making and planning section in the motor system hierarchy whilst, the local leaders correspond to motion generators acting under the influence of the  global leader.

The position of each particle in the population is affected by the influence of global leaders and local leaders, while also being affected by a randomly chosen particle from the population to induce some stochasticity in the optimization pipeline. We first model the influence of the global leader on the local leaders and the influences of the local leaders  on each population element using equation \eqref{one} and \eqref{two}. We introduce a hierarchical crossover between the two influencing equations governed by a hierarchical crossover parameter $HC$.

Analogously to [step 3] in the brain motor operation, the updation of particle positions requires generating feedback for the leaders as a part of the optimization procedure, and hence the local leaders and the global leader are updated based onn their objective function value generated from the perturbations in population particles. This series of events comprise of one optimization pass (one loop step). On execution of several optimization passes as described, the system is able to converge to an optimal configuration, analogous to the successful execution of the required task as shown in [step 4].

The updated position of the particle $x$ is governed by the hierarchical crossover operation and a mutation operation. The hierarchical operation is affected by the global leader $g_L$ and the local leader $l$ through the parametric equations \eqref{one} and \eqref{two}. Switching between the two is governed by the hierarchical crossover parameter $HC$. The given equations are discussed as follows:

\begin{equation}
\label{one}
E_g = g_L + F (l - c)
\end{equation}

\begin{equation}
\label{two}
E_l = l + F (x - c)
\end{equation}

\begin{algorithm}
\caption{Distributed Leader Optimization}
\label{algo}
\begin{algorithmic}[1]
	\Procedure{Start}{}
		\State Initialize parameters ($HC$, $P$, $N_l$, $N$).
		\State Generate initial global leader $g_L$ as a random point.
		\State Generate $N_l$ local leader points around $g_L$.
		\State Using a Normal distribution, generate $N$ points for population $P$ around the local leaders.
		\While{termination criteria is not met}
		%%%%%%%%%%%%%%%%%%%%%%%%%%%%%%%%%%%%%%%%%%%%%%%%%%%%%%%%%%%%%%%%%%%%%%%%%%%%%%%%%%%%%%
			\For{each individual $x_i$ in $P$}
				\State compute the corresponding local leader $l$ based on nearest position.
				\State Let $u = 0$ be an empty vector.
				\State Let $i_c$ and $i_N$ be the current generation and total generations of the procedure.
				\If {$i_c \textless (HC*i_N)$}
					\State $u = E_g$ from \eqref{one}.
				\Else
					\State $u = E_l$ from \eqref{two}.
				\EndIf
				
				\For{each dimension $i$}
					\State Generate $r_i = U(0, 1)$, a random number between 0 and 1.
					\If {$r_i \textless HC$}
						\State Set $x_i^{'} = x_i$.
					\Else
						\State Set $x_i^{'} = u_i$
					\EndIf
				\EndFor
				\If {$f(x^{'}) \textless f(x)$}
					\State Replace $x$ with $x^{'}$ in the population.
				\EndIf
			\EndFor
			\State Alter local leaders in each population cluster based on objective function value.
			\State Compute updated global leader $g_L$.
		\EndWhile
	\EndProcedure
\end{algorithmic}
\end{algorithm}

\begin{figure}
  \includegraphics[scale=0.25]{contourDL}
  \caption{The process for generating generation of $E_g$ and $E_l$ in a 2 dimensional optimization.}
  \label{fig:contourDL}
\end{figure}

In the algorithm \ref{algo}, The Hierarchical crossover is controlled by the conditional equation $i_c \textless (HC*i_N)$. According to this equation, during the initial phases ($HC$ fraction of total generations) of the optimization procedure, only the local leader is responsible for the motion of the agents, and after a certain amount of time has passed, the global leader also takes part in the motion generation process, signifying the motor control operation.
Additionaly, The hierarchical crossover parameter $HC$ also influences the mutation process wherein the degree of final mutation is decided based on the probability $HC$.

\section{Results and Discussions}

All evaluations were performed using Python 2.7.12 with Scipy and Numpy for numerical computations and Matplotlib package for graphical representation of the result data. This section is divided into two sub-sections: Section A provides description about the problem set used for analysis of algorithmic efficiency and accuracy, and section B comprises of tabular and graphical data to support the claim of eminence of the proposed approach.

\subsection{Problem Set Description}

The set of objective functions considered for testing the proposed algorithm and compare it's performance against DE, Particle Swarm Optimization Differential Evolution (PSODE) and Joint Adaptive Differential Evolution (JADE) has been taken from the CEC 2017 [] set of benchmark functions.

\begin{table}[!htbp]
\caption{Test Functions}
\vspace{-3mm}
\centering
%\def\arraystretch{2.0}
\begin{tabular}{|p{0.7cm}|p{5.4cm}|}
\hline
F$_{id}$ & Problem Function \\ \hline
1 & Shifted and Rotated Bent Cigar Function \\
\hline
2 & Shifted and Rotated Sum of Different Power Function \\
\hline
3 & Shifted and Rotated Zakharov Function \\
\hline
4 & Shifted and Rotated Rosenbrock’s Function \\
\hline
5 & Shifted and Rotated Rastrigin’s Function \\
\hline
6 & Shifted and Rotated Expanded Scaffer’s F6 Function \\
\hline
7 & Shifted and Rotated Lunacek Bi\_Rastrigin Function \\
\hline
8 & Shifted and Rotated Non-Continuous Rastrigin’s Function \\
\hline
9 & Shifted and Rotated Levy Function \\
\hline
10 & Shifted and Rotated Schwefel’s Function \\
\hline
11 & Hybrid Function 1 (N=3) \\
\hline
12 & Hybrid Function 2 (N=3) \\
\hline
13 & Hybrid Function 3 (N=3) \\
\hline
14 & Hybrid Function 4 (N=4) \\
\hline
15 & Hybrid Function 5 (N=4) \\
\hline
16 & Hybrid Function 6 (N=4) \\
\hline
17 & Hybrid Function 7 (N=5) \\
\hline
18 & Hybrid Function 8 (N=5) \\
\hline
19 & Hybrid Function 9 (N=5) \\
\hline
20 & Hybrid Function 10 (N=6) \\
\hline
21 & Composition Function 1 (N=3) \\
\hline
22 & Composition Function 2 (N=3) \\
\hline
23 & Composition Function 3 (N=4) \\
\hline
24 & Composition Function 4 (N=4) \\
\hline
25 & Composition Function 5 (N=5) \\
\hline
26 & Composition Function 6 (N=5) \\
\hline
27 & Composition Function 7 (N=6) \\
\hline
28 & Composition Function 8 (N=6)  \\
\hline
29 & Composition Function 9 (N=3)  \\
\hline
30 & Composition Function 10 (N=3) \\
\hline
\multicolumn{2}{|c|}{Search Range: [-100,100]$^{D}$ }  \\
\hline
\end{tabular}
\vspace{-5mm}
\end{table}



\subsection{Results}

% \begin{table*}[!htb]
% \centering
% \caption{}
% \centering
% \begin{tabular}{|l|l|l|l|l|l|l|l|l|l|l|}
% \hline
% \multicolumn{2}{|l|}{DE} & \multicolumn{5}{l|}{PSODE} & \multicolumn{2}{l|}{JADE} & \multicolumn{2}{l|}{Ours} \\ \hline
% $F$  &  $Cr$  &  $w$ & $Cp$ & $Cg$ & $F$  & $Cr$ & $\mu_{CR}$ &  $\mu_{F}$  &  $HC$  &  $n_{leaders}$  \\ \hline
% 0.4 &  0.48 &  0.7  &  2.0 & 2.0 & 0.48 & 0.5 & 0.5  &  0.5   &  0.375  &  5 \\ \hline
% \end{tabular}
% \end{table*}


% Please add the following required packages to your document preamble:
% \usepackage{multirow}
\begin{table}[b]
\centering
\caption{Algorithm Parameter Settings used for comparision}
\label{table:params}
\begin{tabular}{|l|l|l|}
\hline
Algorithm & Parameter & Value \\
\hline
\multirow{2}{*}{DE \cite{storn1995differential},\cite{Mezura-Montes},\cite{brest2006self}} & $F$ & 0.5 \\ \cline{2-3} 
                  & $CR$ & 0.9 \\ \hline
\multirow{5}{*}{PSODE \cite{}} & $w$ & 0.7 \\ \cline{2-3} 
                  & $Cp$ & 2.0 \\ \cline{2-3} 
                  & $Cg$ & 2.0 \\ \cline{2-3} 
                  & $F$ & 0.48 \\ \cline{2-3} 
                  & $CR$ & 0.5 \\ \hline
\multirow{2}{*}{JADE \cite{zhang2009jade}} & $\mu_{CR}$ & 0.5 \\ \cline{2-3} 
                  & $\mu_{F}$ & 0.5 \\ \hline
\multirow{3}{*}{HIDE} & $HC$ & 0.27 \\ \cline{2-3}
				& $F$ & 0.48 \\ \cline{2-3}
				& $CR$ & 0.9 \\ \cline{2-3}
                  & $N_{l}$ & 5 \\ \hline
\end{tabular}
\end{table}

\begin{table*}[t!]
\centering
\caption{Objective Function Value for Dimension: 10}
\resizebox{1.6\columnwidth}{!}{
 \begin{tabular}{|p{0.8cm}|p{1.6cm}|p{1.6cm}|p{1.6cm}|p{1.6cm}|p{1.6cm}|p{1.6cm}|p{1.6cm}|p{1.6cm}|} 
 \hline
 ID & \multicolumn{2}{c|}{DE} & \multicolumn{2}{c|}{JADE} & \multicolumn{2}{c|}{PSO-DE} & \multicolumn{2}{c|}{HIDE} \\
 \hline
    & best & mean & best & mean & best & mean & best & mean \\ [0.5ex] 
 \hline
$f_1$  & 100.000051 & 100.011085 & \textbf{100.0} & \textbf{100.0} & 100.000712 & 185.975885 & \textbf{100.0} & \textbf{100.0} \\ 
 % \hline
$f_2$  & \textbf{200.0} & 200.1 & \textbf{200.0} & \textbf{200.0} & \textbf{200.0} & \textbf{200.0} & \textbf{200.0} & \textbf{200.0} \\ 
 % \hline
$f_3$  & 300.00134 & 300.214502 & \textbf{300.0} & \textbf{300.0} & 300.000006 & 300.000985 & \textbf{300.0} & \textbf{300.0} \\ 
 % \hline
$f_4$  & 400.042617 & 403.674837 & \textbf{400.0} & 400.409399 & 400.064644 & 404.307763 & \textbf{400.0} & \textbf{400.000003} \\ 
 % \hline
$f_5$  & 566.661791 & 604.867489 & \textbf{523.908977} & \textbf{541.521084} & 525.868824 & 575.61616 & 533.803201 & 579.483815 \\ 
 % \hline
$f_6$  & 621.914237 & 634.807962 & 620.878276 & 636.034759 & \textbf{603.187964} & 635.865001 & 613.730565 & \textbf{629.293758} \\ 
 % \hline
$f_7$  & 724.831278 & 739.129935 & \textbf{717.016542} & \textbf{723.983312} & 725.44788 & 733.15638 & 720.345706 & 725.233785 \\ 
 % \hline
$f_8$  & \textbf{818.904202} & 829.749207 & 821.914433 & \textbf{826.321588} & 820.8941 & 830.246691 & 821.064763 & 828.160987 \\ 
 % \hline
$f_9$  & \textbf{900.0} & 908.104383 & \textbf{900.0} & 1084.47825 & \textbf{900.0} & 1124.102561 & \textbf{900.0} & \textbf{903.454324} \\ 
 % \hline
$f_{10}$  & 1911.51009 & 2447.44375 & 1760.95686 & 2162.64858 & 2049.64472 & 2518.24109 & \textbf{1694.43759} & \textbf{2049.07426} \\ 
 % \hline
$f_{11}$  & 1102.98570 & 1113.42310 & 1105.66167 & 1117.50974 & 1105.97013 & 1120.19297 & \textbf{1101.76974} & \textbf{1108.86359} \\ 
 % \hline
$f_{12}$  & 2531.74630 & 6509.74307 & 1438.60571 & 5430.67468 & 4089.00635 & 10810.3876 & \textbf{1308.43834} & \textbf{1327.40588} \\ 
 % \hline
$f_{13}$ & 1313.13022 & 1404.90360 & \textbf{1304.68156} & \textbf{1328.75526} & 1319.83919 & 1453.34078 & 1306.68204 & 1344.28224 \\ 
 % \hline
$f_{14}$  & 1409.94961 & 1426.57193 & 1412.93443 & 1428.16943 & 1420.91065 & 1434.11288 & \textbf{1404.92899} & \textbf{1410.00077} \\ 
 % \hline
$f_{15}$  & 1504.13139 & 1521.44661 & 1502.49618 & 1508.31154 & 1501.38951 & 1518.31035 & \textbf{1500.08137} & \textbf{1503.16926} \\ 
 % \hline
$f_{16}$  & 1958.42062 & 2104.55573 & 1958.85799 & 2094.63082 & \textbf{1958.41153} & \textbf{2048.15688} & 1958.43351 & 2062.38595 \\ 
 % \hline
$f_{17}$  & 1728.19497 & \textbf{1743.15524} & 1730.71532 & 1748.12987 & 1727.80039 & 1791.60774 & \textbf{1723.85397} & 1747.58908 \\ 
 % \hline
$f_{18}$  & 1801.58601 & 1838.84055 & 1804.29854 & 1825.09164 & 1817.15464 & 1840.54692 & \textbf{1800.23551} & \textbf{1804.01430} \\ 
 % \hline
$f_{19}$  & 1901.19548 & 1903.60476 & 1900.39977 & 1902.15296 & 1902.71174 & 1906.25233 & \textbf{1900.00563} & \textbf{1901.01412} \\ 
 % \hline
$f_{20}$  & 2204.55412 & 2289.22658 & 2148.53894 & 2178.31317 & 2140.56131 & 2261.03877 & \textbf{2139.91553} & \textbf{2172.81652} \\ 
 % \hline
$f_{21}$  & 2337.77299 & 2387.23036 & \textbf{2314.42114} & \textbf{2338.68872} & 2337.20734 & 2351.89886 & 2320.49621 & 2344.61612 \\ 
 % \hline
$f_{22}$  & 2300.80585 & 2304.13288 & \textbf{2300.0} & \textbf{2300.09348} & 2300.68418 & 2301.71048 & 2300.00002 & 2301.09598 \\ 
 % \hline
$f_{23}$  & 3070.17708 & 3145.77229 & 3003.67856 & 3091.22041 & \textbf{2773.37286} & 3060.02252 & 2867.02004 & \textbf{3047.98231} \\ 
 % \hline
$f_{24}$  & \textbf{2500.0} & \textbf{2500.0} & \textbf{2500.0} & \textbf{2500.0} & \textbf{2500.0} & \textbf{2500.0} & \textbf{2500.0} & \textbf{2500.0} \\ 
 % \hline
$f_{25}$  & 2899.58497 & 2933.24981 & 2899.58497 & 2930.26651 & \textbf{2897.74287} & \textbf{2921.27479} & 2897.83339 & 2927.97651 \\ 
 % \hline
$f_{26}$  & \textbf{2800.0} & 4117.597033 & \textbf{2800.0} & \textbf{2956.06417} & \textbf{2800.0} & 3367.60765 & \textbf{2800.0} & 3161.54808 \\ 
 % \hline
$f_{27}$  & 3113.15765 & 3358.80643 & 3072.43902 & 3178.50964 & 3078.87313 & 3240.50181 & \textbf{3071.20357} & \textbf{3107.26854} \\ 
% \hline
$f_{28}$  & 3184.75565 & 3230.92142 & 3184.75565 & \textbf{3195.11304} & 3184.75565 & 3198.37069 & \textbf{3100.0} & 3195.41196 \\ 
 % \hline
$f_{29}$  & \textbf{3148.58712} & 3266.97978 & 3172.40019 & \textbf{3233.70768} & 3191.34819 & 3244.89264 & 3189.21142 & 3292.42047 \\ 
 % \hline
$f_{30}$  & 3442.55509 & 11927.40468 & 3207.76694 & 4615.59132 & 4573.35851 & 16415.1629 & \textbf{3205.74095} & \textbf{3249.71098} \\
\hline
$w/t/l$  & 2/4/24 & 1/1/28 & 5/7/18 & 9/4/17 & 4/4/22 & 2/2/26 & 12/7/11 & 14/4/12 \\
\hline
\end{tabular}}
\end{table*}


%\include{table2}
% \begingroup
% \renewcommand\arraystretch{0.7}
\begin{table*}[t!]
\centering
\caption{Objective Function Value for Dimension: 50}
%\vspace{-3mm}
 \begin{tabular}{|p{0.8cm}|p{1.6cm}|p{1.6cm}|p{1.6cm}|p{1.6cm}|p{1.6cm}|p{1.6cm}|p{1.6cm}|p{1.6cm}|} 
 \hline
$f_{id}$ & \multicolumn{2}{c|}{DE} & \multicolumn{2}{c|}{JADE} & \multicolumn{2}{c|}{PSO-DE} & \multicolumn{2}{c|}{HIDE} \\
 \hline
    & best & mean & best & mean & best & mean & best & mean \\ [0.5ex] 
 \hline
$f_{1}$  & 5884574.87 & 367294248.5 & 136.072384 & 3708.75086 & 5811.21899 & 154233.646 & \textbf{106.072862} & \textbf{3665.41927} \\ 
 % \hline
$f_{2}$  & 4.7181e+24 & 3.3649e+44 & \textbf{2635725.0} & \textbf{5.0237e+26} & 2.2121e+19 & 2.5445e+23 & 2.2799e+17 & 1.0072e+31 \\ 
 % \hline
$f_{3}$  & 45520.9663 & 62237.2968 & 143481.793 & 156166.762 & 52308.4274 & 64435.2406 & \textbf{44613.2999} & \textbf{58182.8373} \\
 % \hline
$f_{4}$  & 574.400328 & 801.384952 & 418.580378 & 470.113207 & 477.080964 & 574.528479 & \textbf{400.005049} & \textbf{447.775413} \\ 
 % \hline
$f_{5}$  & 816.394775 & 843.258843 & 809.89948 & 834.13126 & \textbf{778.59312} & 831.066954 & 791.405194 & \textbf{830.218472} \\ 
 % \hline
$f_{6}$  & 652.54191 & 655.794152 & \textbf{633.21788} & \textbf{654.893828} & 653.291336 & 658.183613 & 645.25633 & 656.060597 \\ 
 % \hline
$f_{7}$  & 1109.02123 & 1263.03848 & \textbf{889.036574} & \textbf{944.90319} & 915.153525 & 1047.43879 & 989.957862 & 1186.2487 \\ 
 % \hline
$f_{8}$  & 1139.27892 & 1175.8931 & 1118.3391 & \textbf{1144.60474} & \textbf{1092.62639} & 1159.03235 & 1100.4760 & 1168.5299 \\ 
 % \hline
$f_{9}$  & 22196.3878 & 29218.7759 & 11958.2800 & \textbf{13174.6623} & 24753.0405 & 32233.9545 & \textbf{10251.4763} & 14752.7168 \\ 
 % \hline
$f_{10}$  & 6228.49289 & 7289.18367 & 6054.70769 & 6833.30631 & 6207.79530 & 7055.59523 & \textbf{6050.43437} & \textbf{6609.80456} \\ 
 % \hline
$f_{11}$  & 1170.85860 & 1258.51763 & 1202.69485 & 1232.20426 & 1206.15456 & 1252.93954 & \textbf{1156.4396} & \textbf{1205.2544} \\ 
 % \hline
$f_{12}$  & 677263.079 & 16987989.9 & \textbf{74784.6159} & 530814.648 & 584300.698 & 3448448.79 & 126908.215 & \textbf{494471.075} \\ 
 % \hline
$f_{13}$  & 6005.53530 & 16893.94992 & 2041.48812 & 4332.5945 & 1572.25297 & \textbf{4301.82960} & \textbf{1484.76179} & 7760.05613 \\ 
 % \hline
$f_{14}$  & 38490.5323 & 174367.450 & \textbf{2466.04705} & 238838.470 & 16327.4231 & 67939.0002 & 2967.8184 & \textbf{26290.3161} \\ 
 % \hline
$f_{15}$  & 2278.14122 & 26989.2555 & 13553.0418 & 25636.7696 & 3443.58734 & \textbf{9167.26709} & \textbf{1938.20040} & 14976.7218 \\ 
 % \hline
$f_{16}$  & 2722.02601 & 3176.91690 & \textbf{2345.40070} & \textbf{2916.56101} & 2521.93881 & 3146.04527 & 2436.44933 & 2978.37746 \\ 
 % \hline
$f_{17}$  & 2799.94977 & 3289.61565 & 2568.38357 & 2907.86927 & 2887.28110 & 3236.95792 & \textbf{2561.37030} & \textbf{2874.96503} \\ 
 % \hline
$f_{18}$  & 264037.125 & 872072.477 & 36176.5867 & \textbf{113941.317} & \textbf{26965.2851} & 114846.121 & 260540.781 & 536454.326 \\ 
 % \hline
$f_{19}$  & 10051.9124 & 20380.2571 & 2089.17225 & 7763.17234 & 9905.85082 & 16555.7569 & \textbf{2013.12690} & \textbf{3609.25896} \\ 
 % \hline
$f_{20}$  & 2950.92319 & 3274.33401 & 3041.81309 & 3113.28946 & 2991.58929 & 3361.82394 & \textbf{2495.03177} & \textbf{3080.13747} \\ 
 % \hline
$f_{21}$  & 2596.7256 & 2689.68836 & 2526.19089 & 2597.6771 & 2555.8788 & 2642.38159 & \textbf{2447.75827} & \textbf{2570.91101} \\ 
 % \hline
$f_{22}$  & 9713.99324 & 10803.6537 & 10759.5967 & 11032.8809 & 8918.43626 & 10465.0224 & \textbf{8181.4460} & \textbf{9755.0703} \\ 
 % \hline
$f_{23}$ & 3451.10494 & 4200.17442 & 2971.16064 & 3237.77866 & 2977.55496 & 3490.63975 & \textbf{2851.65025} & \textbf{3162.31362} \\ 
 % \hline
$f_{24}$  & 3434.46502 & 3682.84670 & 3103.95517 & 3185.38267 & \textbf{3036.79960} & \textbf{3158.33050s} & 3136.92774 & 3284.65609 \\ 
 % \hline
$f_{25}$  & 3141.14488 & 3292.30344 & 2931.16295 & 2962.47175 & 2931.92695 & 3008.89535 & \textbf{2931.14231} & \textbf{2954.76783} \\ 
 % \hline
$f_{26}$  & 4906.13284 & 7989.49096 & \textbf{2900.0} & 3346.87403 & 2900.44189 & 3653.75774 & \textbf{2900.0} & \textbf{3262.66849} \\ 
 % \hline
$f_{27}$  & 3200.01070 & 3792.64558 & 3143.03805 & 3184.64635 & 3158.17823 & 3397.13032 & \textbf{3141.01087} & \textbf{3176.01152} \\ 
 % \hline
$f_{28}$  & 3300.01082 & 3431.57091 & \textbf{3240.72586} & \textbf{3288.25303} & 3263.20714 & 3300.25760 & 3243.63199 & 3294.37323 \\ 
 % \hline
$f_{29}$  & 3812.47551 & 4605.34953 & \textbf{3533.94574} & \textbf{3956.83524} & 3955.32453 & 4364.18129 & 3653.67555 & 3966.47195 \\ 
 % \hline
$f_{30}$  & 3673.71196 & 5813.17375 & 3916.72571 & 4869.08933 & 3730.30935 & 5143.07870 & \textbf{3346.48367} & \textbf{4747.88675} \\ 
\hline
$w/t/l$  & 0/0/30 & 0/0/30 & 8/1/21 & 9/0/21 & 4/0/26 & 3/0/27 & 17/1/12 & 18/0/12 \\
\hline
\end{tabular}
\end{table*}

% \begin{table*}[b!]
% \centering
% \caption{Objective Function Value for Dimension: 100}
% %\vspace{-3mm}
%  \begin{tabular}{|p{0.8cm}|p{1.6cm}|p{1.6cm}|p{1.6cm}|p{1.6cm}|p{1.6cm}|p{1.6cm}|p{1.6cm}|p{1.6cm}|} 
% \hline
% $f_{id}$ & \multicolumn{2}{c|}{DE} & \multicolumn{2}{c|}{JADE} & \multicolumn{2}{c|}{PSO-DE} & \multicolumn{2}{c|}{HIDE} \\
% \hline
%     & best & mean & best & mean & best & mean & best & mean \\ [0.5ex] 
% \hline
% $f_{1}$  & 34272128e+2 & 13807281e+3 & 141.263356 & 13516.69893 & 6067123.52 & 29751976.5 & \textbf{122.398748} & \textbf{11708.8236} \\ 
%  % \hline
% $f_{2}$  & 4.196e+84 & 1.547e+112 & 8.737e+74 & 2.543e+87 & \textbf{6.153e+66} & \textbf{3.211e+73} & 3.8835e+80 & 8.8914e+114 \\ 
%  % \hline
% $f_{3}$  & 228808.969 & 262699.687 & 312244.360 & 332179.290 & 241427.723 & 257462.977 & \textbf{220765.083} & \textbf{251901.109} \\ 
%  % \hline
% $f_{4}$  & 1975.65115 & 2752.24606 & 539.386275 & 677.05465 & 777.314462 & 836.965399 & \textbf{531.169819} & \textbf{621.219143} \\ 
%  % \hline
% $f_{5}$  & 1223.53650 & 1286.15333 & 1249.19503 & 1307.11012 & 1248.41013 & 1310.88765 & \textbf{1068.11742} & \textbf{1272.47682} \\ 
%  % \hline
% $f_{6}$  & 651.65013 & 657.84974 & 654.70934 & 659.421427 & 656.87704 & 662.31841 & \textbf{642.33355} & \textbf{654.13275} \\ 
%  % \hline
% $f_{7}$  & 1614.00386 & 1920.79772 & 1367.06653 & 1536.35787 & \textbf{1311.84975} & \textbf{1534.20776} & 1562.37977 & 2076.70250 \\ 
%  % \hline
% $f_{8}$  & 1595.41873 & 1736.36737 & 1672.56784 & 1768.08243 & 1678.12726 & 1761.9405 & \textbf{1293.55211} & \textbf{1592.16298} \\ 
%  % \hline
% $f_{9}$  & 59726.5146 & 71986.0439 & 28906.9090 & 30336.7453 & 63640.3313 & 74961.2209 & \textbf{23466.5750} & \textbf{27067.0295} \\ 
%  % \hline
% $f_{10}$  & 12005.8897 & 14725.3483 & 14227.8019 & 15355.6218 & 12937.0278 & 14972.9507 & \textbf{11153.5868} & \textbf{13298.0921} \\ 
%  % \hline
% $f_{11}$  & 7540.6179 & 11481.2601 & 40447.5486 & 57228.6836 & \textbf{3521.90152} & \textbf{4544.80401} & 5380.43205 & 9916.34769 \\ 
%  % \hline
% $f_{12}$  & 529993877 & 1881773e+3 & 3893556.27 & \textbf{6415173.60} & 26105108.9 & 41876679.1 & \textbf{3680108.18} & 10059039.6 \\ 
%  % \hline
% $f_{13}$  & 7943.9249 & 508209.562 & 4622.69855 & \textbf{8892.77599} & 8246.51529 & 12675.8455 & \textbf{2976.84135} & 11376.9863 \\ 
%  % \hline
% $f_{14}$  & 728122.833 & 1329183.17 & \textbf{132194.795} & \textbf{365560.881} & 548410.338 & 941547.524 & 234045.940 & 867160.306 \\ 
%  % \hline
% $f_{15}$  & 2660.46578 & 181957.060 & \textbf{1799.50650} & 3362.50960 & 1899.07344 & \textbf{2914.44348} & 1976.78912 & 4485.4152 \\ 
%  % \hline
% $f_{16}$  & 4749.25466 & 5847.82673 & 4817.48373 & 5632.3022 & 3852.7000 & 5228.6635 & \textbf{3519.49494} & \textbf{4796.80272} \\ 
%  % \hline
% $f_{17}$  & 4397.49635 & 4958.41818 & 3842.20601 & \textbf{4450.17742} & 3790.72056 & 4730.99458 & \textbf{3582.78588} & 5463.21694 \\ 
%  % \hline
% $f_{18}$  & 1357845.39 & 1938893.27 & \textbf{146426.273} & \textbf{763318.822} & 1004224.20 & 2315010.2 & 631040.146 & 1335739.59 \\ 
%  % \hline
% $f_{19}$  & 2482.1701 & 26455.7069 & 2098.9496 & 4767.52953 & 2263.72515 & 3927.45994 & \textbf{2071.07706} & \textbf{3664.15987} \\ 
%  % \hline
% $f_{20}$  & 4968.49743 & 5436.60405 & 5231.02648 & 5690.74899 & 5109.46056 & 5781.30083 & \textbf{3627.77789} & \textbf{5228.43066} \\ 
%  % \hline
% $f_{21}$  & 3180.74665 & 3355.4783 & 2921.90012 & \textbf{3085.6922} & \textbf{2885.57408} & 3127.35683 & 2926.35039 & 3199.98618 \\ 
%  % \hline
% $f_{22}$  & 17808.8977 & 19562.9866 & 19213.3756 & 20278.9290 & 18695.5223 & 20167.41374 & \textbf{17548.3390} & \textbf{19547.1512} \\ 
%  % \hline
% $f_{23}$  & 4907.51964 & 5819.20786 & \textbf{3352.5569} & 4222.43689 & 3582.04355 & 4779.92124 & 3418.98320 & \textbf{3609.0985} \\ 
%  % \hline
% $f_{24}$  & 5173.24940 & 5946.12042 & 4060.95130 & 4095.42951 & \textbf{3801.36858} & \textbf{4042.42685} & 3998.05402 & 4216.82489 \\ 
%  % \hline
% $f_{25}$  & 4089.11891 & 4548.28576 & \textbf{3153.48541} & \textbf{3236.61784} & 3348.38226 & 3407.52658 & 3176.3038 & 3264.31853 \\ 
%  % \hline
% $f_{26}$  & 8557.49856 & 20159.1145 & 2900.07737 & 11924.79947 & 3021.13602 & 8682.03543 & \textbf{2900.00038} & \textbf{7867.5518} \\ 
%  % \hline
% $f_{27}$  & 3200.02335 & 3772.40915 & \textbf{3194.80921} & 3201.67073 & 3200.02417 & 3494.61813 & 3200.02354 & \textbf{3200.02395} \\ 
%  % \hline
% $f_{28}$  & 4947.74515 & 5948.21315 & \textbf{3295.12291} & \textbf{3340.28038} & 3456.82843 & 3542.57130 & 3300.80769 & 3354.71733 \\ 
%  % \hline
% $f_{29}$  & 6004.77442 & 7090.64254 & 5208.71172 & 5970.62868 & 5462.32863 & 6178.55906 & \textbf{4541.19547} & \textbf{5739.29154} \\ 
%  % \hline
% $f_{30}$  & 7798.10621 & 202435555 & \textbf{3584.97477} & 10674.2173 & 3920.32703 & \textbf{7139.46072} & 3850.31709 & 15318.5546 \\ 
% \hline
% $w/t/l$  & 0/0/30 & 0/0/30 & 8/0/22 & 8/0/22 & 5/0/25 & 6/0/24 & 17/0/13 & 16/0/14 \\
% \hline

%  \end{tabular}
% \end{table*}
% \endgroup

%% \begingroup
% \renewcommand\arraystretch{0.5}
\begin{table*}[b!]
\centering
\caption{Objective Function Value for Dimension: 100}
%\vspace{-3mm}
 \begin{tabular}{|p{0.8cm}|p{1.6cm}|p{1.6cm}|p{1.6cm}|p{1.6cm}|p{1.6cm}|p{1.6cm}|p{1.6cm}|p{1.6cm}|} 
\hline
$f_{id}$ & \multicolumn{2}{c|}{DE} & \multicolumn{2}{c|}{JADE} & \multicolumn{2}{c|}{PSO-DE} & \multicolumn{2}{c|}{HIDE} \\
\hline
    & best & mean & best & mean & best & mean & best & mean \\ [0.5ex] 
\hline
$f_{1}$  & 3427212e+3 & 1380728e+4 & 141.263356 & 13516.69893 & 6067123.52 & 29751976.5 & \textbf{122.398748} & \textbf{11708.8236} \\ 
 % \hline
$f_{2}$  & 4.196e+84 & 1.547e+112 & 8.737e+74 & 2.543e+87 & \textbf{6.153e+66} & \textbf{3.211e+73} & 3.8835e+80 & 8.891e+114 \\ 
 % \hline
$f_{3}$  & 228808.969 & 262699.687 & 312244.360 & 332179.290 & 241427.723 & 257462.977 & \textbf{220765.083} & \textbf{251901.109} \\ 
 % \hline
$f_{4}$  & 1975.65115 & 2752.24606 & 539.386275 & 677.05465 & 777.314462 & 836.965399 & \textbf{531.169819} & \textbf{621.219143} \\ 
 % \hline
$f_{5}$  & 1223.53650 & 1286.15333 & 1249.19503 & 1307.11012 & 1248.41013 & 1310.88765 & \textbf{1068.11742} & \textbf{1272.47682} \\ 
 % \hline
$f_{6}$  & 651.65013 & 657.84974 & 654.70934 & 659.421427 & 656.87704 & 662.31841 & \textbf{642.33355} & \textbf{654.13275} \\ 
 % \hline
$f_{7}$  & 1614.00386 & 1920.79772 & 1367.06653 & 1536.35787 & \textbf{1311.84975} & \textbf{1534.20776} & 1562.37977 & 2076.70250 \\ 
 % \hline
$f_{8}$  & 1595.41873 & 1736.36737 & 1672.56784 & 1768.08243 & 1678.12726 & 1761.9405 & \textbf{1293.55211} & \textbf{1592.16298} \\ 
 % \hline
$f_{9}$  & 59726.5146 & 71986.0439 & 28906.9090 & 30336.7453 & 63640.3313 & 74961.2209 & \textbf{23466.5750} & \textbf{27067.0295} \\ 
 % \hline
$f_{10}$  & 12005.8897 & 14725.3483 & 14227.8019 & 15355.6218 & 12937.0278 & 14972.9507 & \textbf{11153.5868} & \textbf{13298.0921} \\ 
 % \hline
$f_{11}$  & 7540.6179 & 11481.2601 & 40447.5486 & 57228.6836 & \textbf{3521.90152} & \textbf{4544.80401} & 5380.43205 & 9916.34769 \\ 
 % \hline
$f_{12}$  & 529993877 & 1881773e+3 & 3893556.27 & \textbf{6415173.60} & 26105108.9 & 41876679.1 & \textbf{3680108.18} & 10059039.6 \\ 
 % \hline
$f_{13}$  & 7943.9249 & 508209.562 & 4622.69855 & \textbf{8892.77599} & 8246.51529 & 12675.8455 & \textbf{2976.84135} & 11376.9863 \\ 
 % \hline
$f_{14}$  & 728122.833 & 1329183.17 & \textbf{132194.795} & \textbf{365560.881} & 548410.338 & 941547.524 & 234045.940 & 867160.306 \\ 
 % \hline
$f_{15}$  & 2660.46578 & 181957.060 & \textbf{1799.50650} & 3362.50960 & 1899.07344 & \textbf{2914.44348} & 1976.78912 & 4485.4152 \\ 
 % \hline
$f_{16}$  & 4749.25466 & 5847.82673 & 4817.48373 & 5632.3022 & 3852.7000 & 5228.6635 & \textbf{3519.49494} & \textbf{4796.80272} \\ 
 % \hline
$f_{17}$  & 4397.49635 & 4958.41818 & 3842.20601 & \textbf{4450.17742} & 3790.72056 & 4730.99458 & \textbf{3582.78588} & 5463.21694 \\ 
 % \hline
$f_{18}$  & 1357845.39 & 1938893.27 & \textbf{146426.273} & \textbf{763318.822} & 1004224.20 & 2315010.2 & 631040.146 & 1335739.59 \\ 
 % \hline
$f_{19}$  & 2482.1701 & 26455.7069 & 2098.9496 & 4767.52953 & 2263.72515 & 3927.45994 & \textbf{2071.07706} & \textbf{3664.15987} \\ 
 % \hline
$f_{20}$  & 4968.49743 & 5436.60405 & 5231.02648 & 5690.74899 & 5109.46056 & 5781.30083 & \textbf{3627.77789} & \textbf{5228.43066} \\ 
 % \hline
$f_{21}$  & 3180.74665 & 3355.4783 & 2921.90012 & \textbf{3085.6922} & \textbf{2885.57408} & 3127.35683 & 2926.35039 & 3199.98618 \\ 
 % \hline
$f_{22}$  & 17808.8977 & 19562.9866 & 19213.3756 & 20278.9290 & 18695.5223 & 20167.41374 & \textbf{17548.3390} & \textbf{19547.1512} \\ 
 % \hline
$f_{23}$  & 4907.51964 & 5819.20786 & \textbf{3352.5569} & 4222.43689 & 3582.04355 & 4779.92124 & 3418.98320 & \textbf{3609.0985} \\ 
 % \hline
$f_{24}$  & 5173.24940 & 5946.12042 & 4060.95130 & 4095.42951 & \textbf{3801.36858} & \textbf{4042.42685} & 3998.05402 & 4216.82489 \\ 
 % \hline
$f_{25}$  & 4089.11891 & 4548.28576 & \textbf{3153.48541} & \textbf{3236.61784} & 3348.38226 & 3407.52658 & 3176.3038 & 3264.31853 \\ 
 % \hline
$f_{26}$  & 8557.49856 & 20159.1145 & 2900.07737 & 11924.79947 & 3021.13602 & 8682.03543 & \textbf{2900.00038} & \textbf{7867.5518} \\ 
 % \hline
$f_{27}$  & 3200.02335 & 3772.40915 & \textbf{3194.80921} & 3201.67073 & 3200.02417 & 3494.61813 & 3200.02354 & \textbf{3200.02395} \\ 
 % \hline
$f_{28}$  & 4947.74515 & 5948.21315 & \textbf{3295.12291} & \textbf{3340.28038} & 3456.82843 & 3542.57130 & 3300.80769 & 3354.71733 \\ 
 % \hline
$f_{29}$  & 6004.77442 & 7090.64254 & 5208.71172 & 5970.62868 & 5462.32863 & 6178.55906 & \textbf{4541.19547} & \textbf{5739.29154} \\ 
 % \hline
$f_{30}$  & 7798.10621 & 202435555 & \textbf{3584.97477} & 10674.2173 & 3920.32703 & \textbf{7139.46072} & 3850.31709 & 15318.5546 \\ 
\hline
$w/t/l$  & 0/0/30 & 0/0/30 & 8/0/22 & 8/0/22 & 5/0/25 & 6/0/24 & 17/0/13 & 16/0/14 \\
\hline

 \end{tabular}
\end{table*}
% \endgroup


\begin{figure*}[hptb]
    \centering
    \begin{subfigure}[b]{0.30\textwidth}
        % \includegraphics[width=\textwidth,natwidth=800,natheight=600]{F1D10}
        \includegraphics[width=\textwidth,natwidth=800,natheight=600]{F1D10}
        \caption{F1D10}
    \end{subfigure}
    \begin{subfigure}[b]{0.30\textwidth}
        \includegraphics[width=\textwidth,natwidth=800,natheight=600]{F1D30}
        \caption{F1D30}
    \end{subfigure}    
    \begin{subfigure}[b]{0.30\textwidth}
        \includegraphics[width=\textwidth,natwidth=800,natheight=600]{F1D50}
        \caption{F1D10}
    \end{subfigure}

    \begin{subfigure}[b]{0.30\textwidth}
        \includegraphics[width=\textwidth,natwidth=800,natheight=600]{F1D100}
        \caption{F1D30}
    \end{subfigure}    
    \begin{subfigure}[b]{0.30\textwidth}
        \includegraphics[width=\textwidth,natwidth=800,natheight=600]{F5D10}
        \caption{F1D10}
    \end{subfigure}
    \begin{subfigure}[b]{0.30\textwidth}
        \includegraphics[width=\textwidth,natwidth=800,natheight=600]{F12D30}
        \caption{F1D30}
    \end{subfigure}
        
    \begin{subfigure}[b]{0.30\textwidth}
        \includegraphics[width=\textwidth,natwidth=800,natheight=600]{F27D50}
        \caption{F1D10}
    \end{subfigure}
    \begin{subfigure}[b]{0.30\textwidth}
        \includegraphics[width=\textwidth,natwidth=800,natheight=600]{F23D100}
        \caption{F1D30}
    \end{subfigure}

    \caption{Comparision analysis over various functions and dimensions}
    \vspace{-4mm}
\end{figure*}


\section{Conclusion}
%%summarisation of performance on different functions at all dimensions.
\begin{table}[h!]
\centering
 \begin{tabular}{|p{0.9cm}|p{1.0cm}|p{1.0cm}|p{1.0cm}}
\hline
Dim & DE & JADE & PSO-DE \\
\hline
10 & v1 & v2 & v3 \\
\hline
30 & v1 & v2 & v3 \\
\hline
50 & v1 & v2 & v3 \\
\hline
100 & v1 & v2 & v3 \\

\end{tabular}
\end{table} 
Differential Evolution is one of the most popular and widely used evolutionary meta-heuristic for the task of optimization. In this work,  we have proposed a new variant of the same called "Hierarchical Motor Differential Evolution", inspired from the hierarchical structure of the brain motor function. This approach enables the population to follow two distince motion patterns, one governed by their local leaders and one by the global leader. Based on these two influences, the individuals try to achieve the global optimum, and have shown to outperform the algorithms taken under consideration by a appreciable factor, as is clearly depicted through the numerical results and performance plots. however, since the proposed algorithm fizzles on a small fraction of the objective functions, we shall continue our quest to improve it's performance through continous modifications through our future work, and analyse it's performance on several real-world applications.


\vfill
\bibliographystyle{apalike}
{\small
\bibliography{example}}


\section*{\uppercase{Appendix}}

\noindent If any, the appendix should appear directly after the
references without numbering, and not on a new page. To do so please use the following command:
\textit{$\backslash$section*\{APPENDIX\}}

\vfill
\end{document}

